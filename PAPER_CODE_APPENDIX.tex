% =============================================================================
% 论文附录:核心代码片段
% Paper Appendix: Core Code Snippets
% MCM 2026 Problem A - Smartphone Battery SOC Continuous-Time Modeling
% =============================================================================

\documentclass[11pt]{article}
\usepackage[utf8]{inputenc}
\usepackage{ctex}  % 支持中文
\usepackage{listings}
\usepackage{xcolor}
\usepackage{booktabs}
\usepackage{geometry}
\usepackage{amsmath}
\usepackage{hyperref}

\geometry{margin=1in}

% Python代码样式设置
\definecolor{codegreen}{rgb}{0,0.6,0}
\definecolor{codegray}{rgb}{0.5,0.5,0.5}
\definecolor{codepurple}{rgb}{0.58,0,0.82}
\definecolor{backcolour}{rgb}{0.95,0.95,0.92}

\lstdefinestyle{pythonstyle}{
    backgroundcolor=\color{backcolour},
    commentstyle=\color{codegreen},
    keywordstyle=\color{blue},
    numberstyle=\tiny\color{codegray},
    stringstyle=\color{codepurple},
    basicstyle=\ttfamily\footnotesize,
    breakatwhitespace=false,
    breaklines=true,
    captionpos=b,
    keepspaces=true,
    numbers=left,
    numbersep=5pt,
    showspaces=false,
    showstringspaces=false,
    showtabs=false,
    tabsize=4,
    language=Python
}

\lstset{style=pythonstyle}

\title{\textbf{Appendix: Core Code Snippets}\\
\large MCM 2026 Problem A - Smartphone Battery SOC Model}
\author{}
\date{}

\begin{document}

\maketitle

本附录包含MCM 2026问题A(智能手机电池SOC连续时间建模)的核心代码片段。

This appendix contains the core code snippets for MCM 2026 Problem A (Smartphone Battery State-of-Charge Continuous-Time Modeling).

% =============================================================================
\section{核心微分方程模型 (Core ODE Model)}
% =============================================================================

\subsection{SOC微分方程}

核心方程 (Core Equation):
\begin{equation}
\frac{dSOC(t)}{dt} = -\frac{P_{total}(t)}{E_{total}}
\end{equation}

其中 $SOC = E_{remaining} / E_{total}$ 表示能量比值。

\begin{lstlisting}
def dSOC_dt(self, t: float, SOC: float, usage_func=None) -> float:
    """
    Core differential equation (energy-based):
    dSOC(t)/dt = -P_total(t) / E_total
    
    SOC Definition: SOC = E_remaining / E_total
    
    Parameters:
        t : Time in hours
        SOC : State of charge (0 to 1)
        usage_func : Returns UsageState for given time t
    
    Returns:
        Rate of change of SOC (per hour)
    """
    usage = usage_func(t) if usage_func else UsageState()
    
    # Total power in Watts
    P_total_W = self.calculate_P_total(usage) / 1000.0
    
    # Effective energy capacity (Wh): E = V_nominal * Q
    C_eff_Wh = self.get_effective_capacity_Wh(usage.temperature_C)
    
    # Core equation: dSOC/dt = -P_total / E_total
    return -P_total_W / C_eff_Wh
\end{lstlisting}

\subsection{ODE数值求解}

\begin{lstlisting}
def simulate(self, t_span, SOC_initial=1.0, usage_func=None):
    """
    Solve the ODE to simulate battery discharge.
    
    Parameters:
        t_span : (t_start, t_end) in hours
        SOC_initial : Initial SOC (0 to 1)
        usage_func : Time-varying usage function
    """
    def ode(t, y):
        return [self.dSOC_dt(t, y[0], usage_func)]
    
    # Terminal event: shutdown at 5% SOC
    def shutdown(t, y):
        return y[0] - 0.05
    shutdown.terminal = True
    shutdown.direction = -1
    
    sol = solve_ivp(ode, t_span, [SOC_initial], 
                    events=shutdown, dense_output=True)
    return sol.t, sol.y[0]
\end{lstlisting}

% =============================================================================
\section{功耗模型 (Power Consumption Model)}
% =============================================================================

总功耗公式 (Total Power Equation):
\begin{equation}
P_{total} = P_{base} + P_{screen} + P_{cpu} + P_{network} + P_{GPS} + P_{other}
\end{equation}

\subsection{总功耗计算}

\begin{lstlisting}
def calculate_P_total(self, usage: UsageState) -> float:
    """
    Total power: P_total = P_base + P_screen + P_cpu + P_network + P_GPS + P_other
    Returns power in mW.
    """
    P_total = self.power.P_base_mW  # Base power: ~80 mW
    P_total += self.calculate_P_screen(usage)
    P_total += self.calculate_P_cpu(usage)
    P_total += self.calculate_P_network(usage)
    
    if usage.gps_active:
        P_total += self.power.P_gps_mW  # ~350 mW
    if usage.bluetooth_active:
        P_total += self.power.P_bluetooth_mW  # ~15 mW
        
    return P_total
\end{lstlisting}

\subsection{屏幕功耗模型}

屏幕功耗公式 (Screen Power Equation):
\begin{equation}
P_{screen} = k_{screen} \times A \times B(t)^n
\end{equation}

基于 Carroll \& Heiser (2010) 的测量数据:74-412 mW。

\begin{lstlisting}
def calculate_P_screen(self, usage: UsageState) -> float:
    """
    Screen power model: P_screen = k_screen * A * B(t)^n
    Based on Carroll & Heiser (2010): 74-412 mW range
    """
    if not usage.screen_on:
        return 0.0
    
    P_min, P_max = 74, 412  # mW
    B_norm = max(0.01, usage.brightness)
    brightness_factor = B_norm ** 2.0  # Brightness exponent n=2
    
    return P_min + (P_max - P_min) * brightness_factor
\end{lstlisting}

\subsection{CPU功耗模型}

CPU功耗公式 (CPU Power Equation):
\begin{equation}
P_{cpu} = P_{static} + k_{cpu} \times U^m \times f^p
\end{equation}

其中 $m=1.5$, $p=2.8$ (CMOS动态功耗特性),基于 Zhang et al. (2010)。

\begin{lstlisting}
def calculate_P_cpu(self, usage: UsageState) -> float:
    """
    CPU power model: P_cpu = P_static + k_cpu * U^m * f^p
    Based on Zhang et al. (2010): CMOS dynamic power scaling
    """
    U = usage.cpu_utilization        # 0 to 1
    f_norm = usage.cpu_frequency_normalized  # 0 to 1
    
    # P_dynamic ~ U^m * f^p (m=1.5, p=2.8 for CMOS)
    P_dynamic = 0.8 * (U ** 1.5) * (f_norm ** 2.8) * 4000
    
    return 50 + P_dynamic  # P_static = 50 mW
\end{lstlisting}

\subsection{网络功耗模型}

网络功耗公式 (Network Power Equation):
\begin{equation}
P_{network} = P_{idle}(Mode) + k_{tx} \times R_{tx} + k_{rx} \times R_{rx}
\end{equation}

基于 Pathak et al. (2012)。

\begin{lstlisting}
def calculate_P_network(self, usage: UsageState) -> float:
    """
    Network power: P_network = P_idle(Mode) + k_tx*R_tx + k_rx*R_rx
    Based on Pathak et al. (2012)
    """
    P = 0.0
    if usage.wifi_active:
        P += 25 + 0.1*usage.data_rate_tx_kbps + 0.05*usage.data_rate_rx_kbps
    if usage.cellular_active:
        P += 80 + 0.15*usage.data_rate_tx_kbps + 0.08*usage.data_rate_rx_kbps
    return P
\end{lstlisting}

% =============================================================================
\section{电池老化模型 (Battery Aging Model)}
% =============================================================================

\subsection{有效容量计算}

有效能量容量公式 (Effective Energy Capacity):
\begin{equation}
E_{eff} = E_{nominal} \times f_{temp}(T) \times f_{age}(n)
\end{equation}

\begin{lstlisting}
def get_effective_capacity_Wh(self, temperature: float, cycles: int) -> float:
    """
    Effective energy capacity considering temperature and aging:
    E_eff = E_nominal * f_temp(T) * f_age(n)
    """
    # Temperature factor (interpolated from industry data)
    temp_factor = self._get_temp_factor(temperature)
    
    # Aging factor (exponential model from NASA data)
    alpha = 0.0012  # Capacity fade coefficient
    age_factor = max(0.7, np.exp(-alpha * cycles))
    
    # E_nominal = V_nominal * Q_nominal
    E_nominal = 3.85 * 4.5  # V * Ah = Wh
    return E_nominal * temp_factor * age_factor
\end{lstlisting}

\subsection{容量衰减模型(基于NASA数据拟合)}

线性容量衰减模型 (Linear Capacity Fade Model):
\begin{equation}
C(n) = C_0 \times (1 - \gamma \times n)
\end{equation}

其中 $\gamma \approx 0.0008$ (每循环衰减0.08\%)。

\begin{lstlisting}
def estimate_capacity_fade_rate(self):
    """
    Fit linear capacity fade model to NASA battery data:
    C(n) = C0 * (1 - gamma * n)
    
    Returns mean fade rate gamma ~ 0.0008 (0.08% per cycle)
    """
    fade_rates = []
    for battery_id, data in self.battery_data.items():
        if data.n_cycles >= 50:
            def capacity_model(n, C0, gamma):
                return C0 * (1 - gamma * n)
            
            popt, _ = curve_fit(capacity_model, 
                               data.cycle_numbers, 
                               data.capacities,
                               p0=[data.initial_capacity, 0.002])
            fade_rates.append(popt[1])  # gamma
    
    return np.mean(fade_rates)
\end{lstlisting}

% =============================================================================
\section{开路电压模型 (Open Circuit Voltage Model)}
% =============================================================================

\subsection{OCV多项式模型(基于Zenodo数据)}

五阶多项式OCV模型 (5th-order Polynomial OCV Model):
\begin{equation}
V(SOC) = c_0 + c_1 \cdot SOC + c_2 \cdot SOC^2 + c_3 \cdot SOC^3 + c_4 \cdot SOC^4 + c_5 \cdot SOC^5
\end{equation}

\begin{lstlisting}
def get_ocv(self, soc: float) -> float:
    """
    Open Circuit Voltage using Zenodo polynomial coefficients:
    V(SOC) = c0 + c1*SOC + c2*SOC^2 + c3*SOC^3 + c4*SOC^4 + c5*SOC^5
    """
    soc = np.clip(soc, 0.001, 1)
    
    # Zenodo-fitted coefficients
    c = [3.349, 2.441, -9.555, 20.922, -20.325, 7.381]
    
    V = sum(c[i] * soc**i for i in range(6))
    return max(3.0, V)  # V_min = 3.0V
\end{lstlisting}

% =============================================================================
\section{使用场景预测 (Usage Scenario Predictions)}
% =============================================================================

\subsection{电池续航预测}

\begin{lstlisting}
def predict_time_to_empty(self, power_mw, initial_soc=1.0, 
                          soh=1.0, temperature=25.0):
    """Predict time to reach shutdown SOC (5%)"""
    def ode(t, y):
        if y[0] <= 0.05:
            return [0.0]
        E_wh = 3.7 * 4.5 * soh  # Energy capacity
        return [-power_mw/1000 / E_wh]
    
    def shutdown(t, y):
        return y[0] - 0.05
    shutdown.terminal = True
    shutdown.direction = -1
    
    sol = solve_ivp(ode, [0, 100], [initial_soc], events=shutdown)
    return sol.t[-1]

# Example usage scenarios
scenarios = {
    'idle':     {'brightness': 0.0, 'cpu_load': 0.05},  # ~50h
    'light':    {'brightness': 0.3, 'cpu_load': 0.15},  # ~25h
    'moderate': {'brightness': 0.5, 'cpu_load': 0.35},  # ~15h
    'heavy':    {'brightness': 0.7, 'cpu_load': 0.55},  # ~8h
    'gaming':   {'brightness': 0.9, 'cpu_load': 0.85},  # ~4h
}
\end{lstlisting}

% =============================================================================
\section{敏感性分析 (Sensitivity Analysis)}
% =============================================================================

\begin{lstlisting}
def sensitivity_analysis(model, base_params):
    """
    Analyze sensitivity of battery life to each parameter
    """
    results = {}
    base_tte = model.predict_time_to_empty(base_params)
    
    # Brightness sensitivity
    for b in [0.1, 0.3, 0.5, 0.7, 0.9]:
        tte = model.predict(brightness=b)
        results[f'brightness_{b}'] = (tte - base_tte) / base_tte
    
    # CPU load sensitivity
    for cpu in [0.1, 0.3, 0.5, 0.7, 0.9]:
        tte = model.predict(cpu_load=cpu)
        results[f'cpu_{cpu}'] = (tte - base_tte) / base_tte
    
    # SOH (battery aging) sensitivity
    for soh in [1.0, 0.9, 0.8, 0.7]:
        tte = model.predict(soh=soh)
        results[f'soh_{soh}'] = (tte - base_tte) / base_tte
    
    return results
\end{lstlisting}

% =============================================================================
\section{数据来源说明 (Data Sources)}
% =============================================================================

\begin{table}[h]
\centering
\caption{数据来源 (Data Sources)}
\begin{tabular}{lll}
\toprule
\textbf{数据集} & \textbf{来源} & \textbf{用途} \\
\midrule
NASA Battery & NASA Ames Prognostics Repository & 容量衰减参数估计 \\
Zenodo AndroWatts & Zenodo.org/records/14314943 & 功耗分解与亮度-功耗关系 \\
Carroll \& Heiser (2010) & USENIX ATC & 屏幕/CPU基准功耗 \\
Pathak et al. (2012) & EuroSys & 网络功耗模型 \\
Zhang et al. (2010) & CODES+ISSS & CPU功耗公式 \\
\bottomrule
\end{tabular}
\end{table}

% =============================================================================
\section{模型参数汇总 (Model Parameters Summary)}
% =============================================================================

\begin{table}[h]
\centering
\caption{模型参数 (Model Parameters)}
\begin{tabular}{llll}
\toprule
\textbf{参数} & \textbf{符号} & \textbf{值} & \textbf{来源} \\
\midrule
标称容量 & $C_{nominal}$ & 4500 mAh & Modern smartphone \\
标称电压 & $V_{nominal}$ & 3.85 V & Li-ion standard \\
容量衰减率 & $\gamma$ & 0.0008/cycle & NASA data fitting \\
基础功耗 & $P_{base}$ & 80 mW & Literature \\
屏幕功耗 & $P_{screen}$ & 74-412 mW & Carroll \& Heiser \\
CPU静态功耗 & $P_{cpu,static}$ & 50 mW & Zhang et al. \\
关机阈值 & $SOC_{shutdown}$ & 5\% & BMS standard \\
\bottomrule
\end{tabular}
\end{table}

\end{document}
